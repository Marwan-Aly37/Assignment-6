\documentclass{article} 
 
\usepackage[utf8]{inputenc} 
\usepackage{lipsum} % For generating dummy text 
\usepackage{graphicx} % For including images 
\usepackage{url} % For including URLs 
\usepackage{hyperref} % For hyperlinking within the document 
 
\title{Path Planning using \LaTeX} 
\author{Your Name} 
\date{\today} 
 
\begin{document} 
 
\maketitle 
 
\section{Introduction to Path Planning} 
Path planning is a vital component of various applications, including robotics, autonomous vehicles, and computer games. It involves finding an optimal path from a starting point to a goal while avoiding obstacles. Path planning algorithms play a crucial role in enabling autonomous systems to navigate complex environments efficiently and safely. 
 
\section{Types of Path Planning Algorithms} 
There are several types of path planning algorithms, each with its own characteristics and areas of application. Here are some commonly used algorithms: 
\subsection{Dijkstra's Algorithm} 
Dijkstra's algorithm is a well-known algorithm used to find the shortest path in a graph. It explores all possible paths from the starting point to the goal and selects the one with the minimum cost. This algorithm guarantees optimality but may be computationally expensive for large-scale problems. 
 
\subsection{A* Algorithm} 
The A* algorithm is an extension of Dijkstra's algorithm that incorporates heuristics to guide the search towards the goal more efficiently. It uses a combination of the cost to reach a node and an estimate of the remaining cost to the goal to determine the next best node to explore. A* is widely used in robotics and video games due to its efficiency. 
 
\subsection{Rapidly-exploring Random Trees (RRT)} 
Rapidly-exploring Random Trees (RRT) is a sampling-based algorithm that constructs a tree by iteratively exploring the state space. It randomly generates new states and connects them to the existing tree, gradually expanding towards the goal. RRTs are particularly effective for high-dimensional and continuous state spaces. 
 
\subsection{Probabilistic Roadmaps (PRM)} 
Probabilistic Roadmaps (PRM) is another sampling-based algorithm that builds a graph representation of the environment. It randomly samples valid configurations and connects them based on certain criteria, forming a roadmap. PRM can handle complex environments but may require significant pre-processing time. 
 
\section{Local Planner and Global Planner} 
Path planning can be divided into two main components: local planner and global planner. 
\subsection{Local Planner} 
The local planner focuses on short-term decisions for navigating around immediate obstacles. It takes into account the current state and surroundings of the robot or vehicle and generates control actions to avoid collisions. Local planners are typically reactive and operate in real-time. 
 
\subsection{Global Planner} 
The global planner considers the entire environment and plans a path from the starting point to the goal. It utilizes path planning algorithms to find an optimal or near-optimal path while avoiding obstacles. Global planners are responsible for long-term navigation decisions and typically operate at a lower frequency than local planners. 
 
\section{Challenges and Future Trends in Path Planning} 
Despite significant advancements in path planning, there are still challenges that researchers and engineers are working to address. Some of these challenges include: 
 
\subsection{High-dimensional State Spaces} 
Path planning becomes increasingly challenging in high-dimensional state spaces, such as those encountered in robotics or complex virtual environments. Developing efficient algorithms for such spaces remains an active area of research. 
 
\subsection{Real-time Planning} 
Real-time planning is crucial for applications like autonomous vehicles, where decisions must be made quickly. Balancing the need for optimality with computational efficiency is an ongoing challenge. 
 
\subsection{Uncertainty and Dynamic Environments} 
Path planning algorithms need to account for uncertainties